\documentclass{article}
\usepackage{graphicx}
\usepackage{natbib}
\usepackage{amsmath}

\begin{document}
\section{Introduction}
%ms ich würde hier gleich zum Punkt kommen
Recently, Gao and Scholl presented a set of studies \citep{gao09,gao10,gao11} demonstrating that certain aspects of motion influence the detection of chasing and performance in interactive tasks where subjects controlled chasee with mouse. 
In \citet{gao10} the authors focused on the contribution of agent's orientation. 
They found that visual scenarios where multiple moving geometrical shapes were oriented towards a common target (prey) were more difficult than control conditions where agents' orientation was shifted by 90 degrees. 
For example, in Experiment 2 the subjects controlled a green disc with computer mouse and tried to avoid contact with a white circle which chased the green circle. 
The display included six other randomly moving white circles which served as distractors and made the task difficult. 
As a manipulation the authors added seven white darts that were in half of the trials oriented towards the green circle and in other half perpendicular to it. 
%ms hier vielleicht ein Bild das zeigt wie man die Ausrichtung manipuliert? 
They compared across the two conditions the proportion of trials where chaser caught the green circle (relative to trials where he didn't). 
Even though the subjects were explicitly told to ignore the darts, they were worse at escape in trials where the darts pointed toward green circle. 
In Experiment 4 the chaser was the only circle in the display, so it was easily identified. 
However the subjects needed to additionaly avoid contact with darts. 
Escape rate was lower in trials where the darts were oriented towards green circle than in trials where the orientation was perpendicular. 
In Experiment 3a the comparison was not between trials, but rather the display was divided into areas which contained darts with different orientation. 
Subjects' task was to avoid contact with agents. 
Authors found that in doing so the subjects spent more time in areas with darts oriented perpendicular to the green circle than in areas where darts pointed directly towards it.
The authors named the consistent influence of orientation, the wolfpack effect.\\
If we want to conclude that the mentioned effects are due to orientation, one additional assumption is crucial. 
The experimentally manipulated orientation change has to be perceived as an orientation change and not as displacement. 
This is illustrated in figure \ref{fig.dart} on the example of darts used by \citet{gao10}. 
The authors designated the concave vertex (red) as the nominal point for rotation. 
Intuitively however, the center of mass (blue) looks like a much better choice. 
What happens if the perceived center of rotation is displaced in the direction of center of mass.
In this case the rotation around the nominal center is perceived as motion towards chasee and we can contrive an alternative explanation for the results in \citet{gao10}. 
In Experiment 3a the subjects avoided areas with darts oriented towards the prey because the agents' positions were perceived as shifting closer to the green circle. 
In Experiment 4 if we assume that subjects estimated the critical distance to surrounding agents and based their decisions where to move the green circle on this estimate. 
If the perceived agents' location was shifted towards chasee in wolfpack trials, the subjects would often prematurely leave the current location in exchange for another location which would be a bad choice from the point of view of the nominal dart's center which was used to decide whether the green circle was caught or not (see Figure \ref{fig.exp4}). 
Results of Experiment 3a and 4 can thus be alternatively interpreted in terms of domain general processes such as distance estimation and decision making. 
It's hard to come up with domain-general explanation for results of Experiment 2 (and 1). 
Still, since the the intended orientation change is perceived as motion, the results can be explained by the influence of motion cues on subjects judgment.\\
Experiment 3b in \citet{gao10} was the only experiment that did not use darts. 
The design and results were similar to that of Experiment 3a, except that instead of darts the subject tried to avoid white circles whose orientation was determined by two red dots ('eyes', see figure \ref{fig.eyes}).
We will refer to these stimuli as bugs. 
If we use the shape of the agent to determine the center of mass, this is identical to the point around which the circle was rotated. 
Still, this doesn't mean that the bug's perceived position isn't influenced by the orientation of its eyes.\\
The literature on memory displacement is relevant here \citep{hubbard05}. 
It has been repeatedly demonstrated that if subjects are asked to give the last position of a stimulus that just disappeared from the screen, subject's responses are systematically displaced by factors such as gravity, momentum or shape. 
Crucially, some studies \citep{freyd92}[e.g.] demonstrated that displacement can also be influenced by information about stimulus animacy. 
The memory displacement is studied by asking subjects to locate the last position of an object that just disappeared.
This is usually done by asking subjects to select this position with mouse or alternatively to by selecting which of two probes is nearer to the location where the object disappeared.
How are such displacements in explicit recall relevant to the implicit interactive tasks used in \citet{gao10}.
According to \citet{hubbard05}[p. 844] "displacement occurs because it aids in the spatial localization of physical objects and facilitates rapid motor responding to objects in the environment". 
He further adds (cf.) that "accurate spatial localization is important for calibrating an observer's response to a stimulus so that a maximally effective and adaptive interaction with that stimulus might be achieved". 
For instance, the displacement due to representational momentum anticipates future position and allows to correct the discrepancy between the position when the action is programmed and when it is performed. 
We can apply this idea to stimuli and tasks in \citet{gao10}. Since agent's head/nose orientation usually predicts the subsequent motion, orientation may be used to predict future position. 
In Experiment 3b it is important for the subject to accurately predict the direction of the object's motion. 
Hence we expect to find memory displacement which is influenced by factors that provide cue to the future motion direction. 
Since biological organisms usually move in direction in which their body and eyes are oriented, agent's orientation is a good candidate for such a cue.  
Due to the perceived displacement the subject would avoid the wolfpack areas because the wolfpack stimuli are perceived as closing in on the green circle.\\
In the current study we investigate the influence of perceived displacement on subject's performance in the interactive tasks from \citet{gao10}. 
In particular we chose to revisit their task from Experiment 3, the so-called leave-me-alone task in which subjects avoided contact with randomly moving objects in a display separated into areas with different orientation. 
We chose this Experiment since it was the only one that demonstrated the wolfpack effect with bug stimuli (displayed in figure \ref{fig.eyes}) where there is no obvious reason for discrepancy between the nominal and perceived object's center (as is the case darts and their shifted center of mass).\\
Our argument can be distilled into two separate claims:
\begin{enumerate}
  \item The perceived center of the dart and bug stimuli is displaced in direction of it's nose/eyes.
  \item This displacement influences subjects' performance in the leave-me-alone task. 
In particular, the subjects avoid wolfpack areas because the manipulated orientation in the wolfpack areas is perceived as motion towards chasee.
\end{enumerate}  

To test the second claim we shift the rotation center along the anteroposterior axis of the agent and measure how this manipulation influences subject's avoidance of wolfpack areas. 
We utilize two tasks to test the first claim. 
First, we append a location recall task to the leave-me-alone task. 
After each trial of leave-me-alone task one dart/bug disappears while all the remaining objects freezee at their last position. 
The subject is asked to select the position where the agent disappeared.
Second, we devised a novel distance bisection task. 
Subjects are shown the same motion as in the leave-me-alone task. 
However only two agents are displayed. 
One agent is oriented towards chasee and the other one is oriented perpendicular to it. 
Subjects are asked to move a green circle so that it remains on the shortest line between the two agents approximately equidistant to both. 
We expect that the locations chosen by subjects will be displaced in direction away from the wolf. 
The use of multiple tasks with different task demands allows us to identify the perceived displacement more reliably. 
Subjects are not oblivious to the displacement in their explicit judgments and can compensate for the discrepancies. %ms citation needed
The distance bisection task avoids these pitfalls.
The task requires continuous real-time adaption to the changing motion pattern of the two agents. 
There is little opportunity to monitor, reflect and correct the judgment. 
Furthermore, one may object that an observed displacement in location recall task may be due to post-perceptual shift in the memory and is not necessarily relevant to the leave-me-alone task. 
The distance bisection task is not susceptible to this objection.\\
On the other hand, the connection between the distance bisection and leave-me-alone task may not be obvious. 
We suggest that subjects estimate the distance to surrounding agents and then based on this information decide where to move next. 
However, the are many more agents displayed in the leave-me-alone task and the relevant ones are mostly much closer than the average agent in the bisection task. 
In this respect, the location recall judgments are more direct evidence that displacement is also perceived during the leave-me-alone task. \\
So far we framed our study as an attempt to investigate influence of a potential confound in previously published study. 
Such confound would seriously undermine the conclusion of this study, notably the claim that agent's orientation is important cue to detection of chasing and to the perception of goal-directed motion more generally. %ms hier kommen noch ein Paar Referenzen..
However, the current study is also of interest to researchers studying displacement in recall and its relation to action. 
If our claims are correct, this would provide another demonstration of a task where the perceived displacement is relevant to the action-based performance in the task. 
Furthermore we can compare the size of the effects across tasks to see how the size of the reported displacement varies with different task demands.
The outline of this report is as follows. %ms kommt wenn ich das fertig habe

%ms soll ich in der Einfuhrung auch die letzte Aufgabe mit statischen Stimuli erwahnen?
%ms bezeichnungen agent und object fur die stimuli konnte man vielleicht vereinhaltlichen so das nur eins davon verwendet wird, green circle und chasee werden auch austauchbar verwendet...   

\section{General Methods}

\subsection{Stimulus}
The Stimulus programming followed the description in \citet{gao10} and the examples provided on Brian Scholls web-page. 
The display was divided into four quadrants, each forming a square of size 5.9 degrees. 
Three white objects (1.9 degrees in diameter) moved inside of each quadrant. 
These objects were either bugs (Experiment 1) or darts (Experiment 2). 
A bug was a circle that had two red eyes drawn on top of it. 
(These were two red circles of size 0.19 degrees, located 0.71 degrees from the center of the white circle and 0.49 degress apart.) 
The dart was a white polygon with four vertices. 
Three of them were located at the perimeter and the remaining vertex was at the origin of the circle. 
The wings of the dart formed an angle of 120 degrees.
The subject steered a a green circle (chasee, 1.2 diameter) with computer mouse. 
The movement was confined to lie within a circular area with 11.75 degrees radius. 
The boundary of the circular area was shown on the screen as a thin gray line. 
The motion of the white objects was generated as follows. 
The objects moved at constant speed of 7.8 degrees per second. 
Each object changed its direction at random intervals with 3 direction changes per second on average. 
The new direction was chosen randomly from a range of -45 to 45 degrees around its current direction. 
The objects were pervasive to each other and upon touching the wall of the designated quadrant they bounced off. \\
The orientation was manipulated as follows. 
Agents were oriented either with their eyes/nose directed towards the agent's location or perpendicular to the agent. 
There were two quadrants with perpendicular orientation and two quandrants with head-on oriented agents (wolfpack). 
The type of the quadrant was chosen randomly on each trial. \\

\subsection{Task and Design}
We tested the bug and dart stimuli in two separate session with different subject samples. 
We refer to these as Experiment 1 and Experiment 2 respectively.
An experiment session consisted of 3 blocks with 42 trials per block.\\
Block 1 went as follows. 
At the start of each trial the circular boundary and the initial position of the green circle was displayed. 
Subject initiated the trial by bringing a cross-hair mouse cursor to the green circle and by clicking on its surface. 
Furthermore, we ensured that the agent's initial position was at least 4 degrees away from the nearest agent. 
Then the agents appeared and the subject tried to avoid contact by moving the green circles within the circular boundary. 
This lasted 17 seconds. 
We appended a location judgment task at the end of the trial. 
After the movement ended the agents and chasee remained stationary. 
One agent disappeared. 
The subject then clicked the last location where he saw the missing agent with a cross-hair cursor. 
The missing agent was chosen randomly from the three agents nearest to chasee. 
Whenever possible we tried to select an agent that did not overlap and so that perpendicular and wolfpack agents were chosen approximately equally often within each block. \\
In the second block the rotation point was shifted along anteroposterior axis. The magnitude of this shift was varied across trials. We used trial with displacement of -0.15, -0.05, 0.05, 0.1, 0.15 and 0.2. This range was suggested by pilot studies. Later we adapted the range and used 0.05, 0.1,0.15, 0.2 0.25 and 0.3. Otherwise the trials were identical: The displacement was either perpendicular or towards the green circle. The circles were organized into four quadrants based on the type of the displacement. Finally, each trial was followed by a location judgment task. This allowed us to check whether the displacement manipulation was succesful. In addition to the location judgments in the first two blocks we measured the time spent by the green circle in the quadrants with respect to the orientation of the circles.\\
In third block only two agents from two different quadrants were shown: one with perpendicular and one with head-on orientation. The initial position of green circle set at the mid-point on the line connecting the two white circles. The trial started after the subjects clicked the green circle. The subjects were asked to move the green circle such that they stay at the mid-point between the two agents. In the first 18 trials the orientation was indicated by eyes. In the remaining 24 trials the eyes were not displayed and instead the agents were physically displaced in the direction of agent's orientation. The magnitude of displacement was varied with manipulated values identical to those used in block 2.\\
The third block concluded with 13 trials. A white circle was shown for random interval of 2-3 seconds. Then the crosshair appeared and the the circle either disappeared (in the first five trials) or it remained visible. The subjects were asked to select the position of the white circle with cross hair. On first ten trials the eyes were displayed and their orientation was chosen randomly. \\
 
\subsection{Eyetracking}
Immediately before taking part in the current experiment the subjects participated as an adult control group in an infant eye-tracking study. 
This experiment took 5-10 minutes. 
Since the subjects were already seated at a calibrated eyetracker we decided to include eyetracking measurements although no analyses of the eyetracking data were planned and none were performed.

\subsection{Procedure}
In block 1 and 2 subjects were instructed to avoid collisions with white agents by moving the green circle with mouse. 
In block 3 subjects were asked to move the green circle such that it stays between the two white agents. \\
The subjects were seated 50-70 cm away from screen (all specifications in degrees of visual angle are referenced to 50 cm distance) of a Tobii T60 display with built-in remote eye-tracker. 
Upon the conclusion of the experiment the subjects were debriefed and dismissed. 
Each block lasted 15 minutes and there was a brief break between them.\\
The experiment was presented and controlled with PsychoPy 1.77 \citep{peirce07} and Tobii SDK 3.0. 
The analyses were performed with Python 2.7 and STAN 2.0.1.

\subsection{Materials and Data}
Subjects gave informed consent to participate in the study and were further given an option to make their data available for sharing with other researchers. 
All subjects agreed. 
The materials and data are available from \url{http://github.com/simkovic/wolfpackRevisited}. 
We  documented all analyses we made in the form of IPython Notebooks which are also available from the online repository. 

\subsection{Analyses}
In block 1 and 2, we are interested in the direction of the displacement of the last location judgments of the missing circle with respect to its actual position. We expect four predictors to influence the displacement. First, representational gravity pulls objects down along the vertical axis. Second, under the influence of representational momentum subjects interpolate future positions along the motion trajectory of object and recall these as the last position. Third, we call the displacement in the direction of the circles orientation (as indicated by the eyes' location or physical displacement in the third block), orientation displacement. Finally, we expect that the last position of the green circle (controled by mouse) will influence the judgment. In particular we expect the judgment to be pulled towards the last mouse position. We call this effect the lazy-hand gravity.\\
More formally, let $\mathbf \theta_i =[\theta_{ix}, \theta_{iy}]$ be the vector of the difference between the indicated and the actual position of the vanished circle at trial $i$ along the two axes. Let $\phi_{ik}$ be the angle given by the direction of the predictor at trial $i$, with $k$ the predictor (as listed in the previous paragraph). Then we are interested in the estimates of the regression coefficients $\alpha^1_k$ given by the equation\\

\begin{equation}
\mathbf \theta_i \sim  \mathcal{N}\left(\sum_k \alpha_k \begin{bmatrix} \cos \phi_{ik} \\ \sin \phi_{ik} \end{bmatrix}, I\sigma_\theta^{-2} \right)
\end{equation} 

As in \citet{gao10} we compute the proportion of the time spent in the wolfpack quadrants on each trial. We then estimate average proportion for each subject. \\
In block 3 we estimate the displacement as follows. We rotate subjects' judgments in each around the nominal mid-point such that the wolf is located at 180 degrees and perpendicular agent is at 0 degrees. On each trial we discard first 2 seconds and compute the mean displacement during the remaining 15 seconds. We then use this data to estimate the displacement across trials and across subjects.\\
Our general strategy is to first estimate parameters for each subject individually and then to pool estimates with a hierarchical model whenever meaningful. Our within-subject design does not allow us to collect more data points so that in some cases the individual estimates are as robust as we would desire. Hierarchical modeling allows us to pool information across subjects and to obtain more reliable estimates of the individual parameters. More information on hierarchical modeling can be found in \citet{gelman07} and \citet{lee11}.\\

\end{document}

